\documentclass{article}
\usepackage[default=sc]{multilang-fonts}

\begin{document}

English text with \zhcn{简体中文 \textbf{简体中文}} and \ja{日本語} inline.

% Or use the longer form
Some \japanese{こんにちは} and \korean{안녕하세요} text.

try furigana: \furigana{骨}{ほね}; try ruby: \ruby{骨}{ほね}; try pinyin: \pinyinx{骨}{hú}


\begin{SCtext}
    默认简体中文 骨,\textbf{骨}
\end{SCtext}

\begin{TCtext}
    默认繁体中文 骨,\textbf{骨}
\end{TCtext}

\begin{HKtext}
    默认香港中文 骨,\textbf{骨}
\end{HKtext}

\begin{JPtext}
    默认日文 骨,\textbf{骨}
\end{JPtext}

\begin{KRtext}
    默认韩文 骨,\textbf{骨}
\end{KRtext}


\begin{itemize}
    \item \texttt{sc:} {\langSC 骨, 為, 戶, 衞, 敍, 佔, 今, 換, 醜, 悉, 習, 躺, 聶, 釀, 夢, 進, 羞, 務, 充, 卸, 換, 誕, 豬, 啟, 汙}
    \item \texttt{tc:} {\langTC 骨, 為, 戶, 衞, 敍, 佔, 今, 換, 醜, 悉, 習, 躺, 聶, 釀, 夢, 進, 羞, 務, 充, 卸, 換, 誕, 豬, 啟, 汙}
    \item \texttt{hk:} {\langHK 骨, 為, 戶, 衞, 敍, 佔, 今, 換, 醜, 悉, 習, 躺, 聶, 釀, 夢, 進, 羞, 務, 充, 卸, 換, 誕, 豬, 啟, 汙}
    \item \texttt{jp:} {\langJP 骨, 為, 戶, 衞, 敍, 佔, 今, 換, 醜, 悉, 習, 躺, 聶, 釀, 夢, 進, 羞, 務, 充, 卸, 換, 誕, 豬, 啟, 汙}
\end{itemize}


\end{document}
